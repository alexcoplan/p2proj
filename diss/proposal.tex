%% TODO: evaluation: refer to technique in success criteria

% Note: this file can be compiled on its own, but is also included by
% diss.tex (using the docmute.sty package to ignore the preamble)
\documentclass[12pt,a4paper,twoside]{article}
\usepackage[UKenglish]{isodate}
\usepackage[pdfborder={0 0 0}]{hyperref}
\usepackage[margin=25mm]{geometry}
\usepackage{graphicx}
\usepackage{parskip}
\usepackage{enumitem}
\usepackage{microtype}
\usepackage{longtable} % so we can break the timetable across two pages

\usepackage[style=numeric,backend=bibtex]{biblatex}
\bibliography{refs.bib}

\begin{document}

\cleanlookdateon

\begin{center}
\Large
Computer Science Tripos -- Part II -- Project Proposal\\[4mm]
\LARGE
A Comparison of Statistical Models and Recurrent Neural Networks for the
Generation of Music\\[4mm]

\large
Alex Coplan, St Catharine's College

Originator: Alex Coplan

\today
\end{center}

\vspace{5mm}

\textbf{Project Supervisor:} Matthew Ireland (mti20)

\textbf{Director of Studies:} Dr S. Taraskin (snt1000)

\textbf{Project Overseers:} Prof.\ M. Fiore \& Prof.\ I. Leslie

\section{Introduction}

The goal of this project is to implement, evaluate, and compare two different
techniques for the algorithmic generation of music. I am particularly interested
in the generation of melody, and ultimately, \emph{polyphony}: multiple
independent melodies interacting with each other in harmonic coherence.  The two
classes of techniques I intend to consider for this project are:
\begin{itemize}[itemsep=0mm]
	\item Statistical history models such as \emph{multiple viewpoint
			systems} \cite{conklin1995viewpoints}.
	\item Recurrent neural networks.
\end{itemize}

The exact statistical model(s) to be investigated will be chosen during and
determined by the end of the research phase of the project.

Whilst the primary aim of this project is a scientific comparison of two
techniques, the implementation itself will be of use to two main classes of
user:
\begin{enumerate}
  \item Composers. The resulting implementation could be integrated into a
    notation package to enable \emph{machine-assisted composition}, where the
    computer acts as a source of inspiration: extending and/or
    generating music for the composer.
  \item Musicologists, who could gain insight into the works of a composer or
    period by exploring the output of one of the models trained on an
    appropriate corpus.
    
\end{enumerate}

The first problem that needs to be tackled, which will make up the core of the
project, is that of melody generation. Once this problem has been addressed, an
investigation into the generation of polyphonic music could form an extension
component. Specifically, one could consider the problem ``given a melody (the
\emph{subject}), compose a second (independent) melody (the
\emph{countersubject}) which interacts with, and is harmonically coherent with
the subject.'' 

I shall follow the approach often taken in the literature of restricting the
domain of source material to stem from a particular musical idiom, e.g.\
\cite{pearce2001evaluation}. This is desirable for a number of reasons, not
least because it introduces a useful evaluation criterion: do the compositions
produced by the system exhibit a coherent musical style, consistent with that
exhibited by the material in the corpus?

In order for this project to be evaluated effectively, in addition to any
information-theoretic or music-theoretic analyses, it is necessary to perform
listening trials with human participants. Pearce et al.\
\cite{pearce2001evaluation} outline a framework for evaluation which allows
empirically-falsifiable claims to be made as a result of the evaluation process.
Evaluation would be performed in the form of a blind trial where (e.g.) the
subjects are asked to classify compositions as human or machine-composed. In
this work, it is noted that the participants exhibited a bias towards
classifying compositions as machine composed. This is something that should be
taken into account when designing the evaluation methodology. An avenue for
investigation in this respect is the method of \emph{three-alternative forced
  choice}.

Conklin \cite{conklin2003music} notes that random walk is not necessarily the
best method of sampling from a statistical distribution such as that of a
multiple viewpoint system (MVS) or Markov chain, since it can lead to output
solutions with overall low probability. In this project, I would therefore also
consider exploring different techniques for sampling from statistical models.
 
\subsection{Background}

Markov processes are natural statistical models for the analysis of melody, and
are well known as tools for composition \cite{ames1989markov}. Although
effective, Markov processes are far from perfect tools for modelling music.
Specifically, a basic pitch-duration Markov process disregards a considerable
amount of musical information available in the context
\cite{conklin1995viewpoints}.  

However, simply incorporating more musical features into the state space of a
Markov chain leads to an exponential blow-up in space complexity and
necessitates both a large amount of training data for good performance, as well
as solving the sparse data problem (\cite{conklin2003music}, section 2.1).
Moreover, Markov chains do not make use of the long-term context of a system,
which is necessary for modelling the broader sense of sequence and structure
which is present in the majority of music.

Conklin et al.\ \cite{conklin1995viewpoints} introduce the method of
\emph{multiple viewpoints} which uses the interpolation of the predictions of
many different context models, each of which considers a different musical
attribute (or some combination of attributes). These include both short-term and
long-term attributes, enabling this method to capture sequence and structure.

It is well known that Recurrent Neural Networks (RNNs) can effectively learn and
generate sequences \cite{graves2013generating}. RNNs have seen more successful
application in music following the introduction of long short-term memory (LSTM)
techniques \cite{eck2002lstm}.  Without use of LSTM, RNNs exhibit similar
problems to Markov chains in that the output does not contain the elements of
musical sequence and structure that one might expect from compositions in the
corpus.

\section{Starting point}

In Lent term of 2016, I gave a talk for a university society on melody
generation using Markov chains. I also constructed a demo in Ruby which
implemented a parser for ABC notation\footnote{\url{http://abcnotation.com/}}
along with a simple Markov chain model, trained of a small corpus of hymn tunes,
which generated tunes by random walk. 

Although this experience led me to this choice of project, the implementation of
a model such as a MVS is considerably more involved, and
the architecture vastly different. The implementation of this model will
therefore be carried out from scratch.  The neural network will be implemented
using a library such as Google's
TensorFlow\footnote{\url{https://www.tensorflow.org/}}.

As an organ scholar (and previously an A-Level music student), I have
considerable experience with performing polyphonic music (and some experience of
analysis), especially that of the renaissance and baroque eras. I believe this
domain knowledge will prove especially useful for making musically-informed
decisions in this project. 

\section{Resources required}

For this project I shall primarily use my own laptop. Git will be used for
version control, with the repository stored on GitHub. I will perform backups of
the project files to an external hard drive as well as the MCS. Should my main
computer suddenly fail, I can easily continue the project using MCS computers by
cloning the code from the GitHub repository.

Although datasets can easily be compiled from online sources, it may also be of
use to have a MIDI keyboard to be able to input arbitrary musical data. I own a
MIDI keyboard which would be suitable for these purposes. Should this fail, I
have access to suitable College equipment. I will make use of open-source
software (such as MuseScore\footnote{\url{https://musescore.org/}}) for
synthesis of MIDI and other musical data. I require no other special resources.

\section{Work to be done}

I will employ an Agile software development methodology when undertaking this
project. The ordered list of sub-tasks within this project are:
\begin{enumerate}

\item Researching the techniques involved and gaining familiarity with
  core libraries.

\item Devising and implementing an internal representation of musical data,
	along with a simple ``music theory engine'' to process this data.  

\item Implementing a simple parser for some form of input notation (ABC, MIDI,
  MusicXML); the exact form of which to be determined in the research phase.  

\item Implementing and iteratively refining the statistical model (e.g.\
  multiple viewpoint system).

\item Implementing and iteratively refining the RNN for melody generation.  

\item Designing and carrying out schemes for both human and quantitative
  evaluation.

\end{enumerate}

\section{Success criteria}

\subsection{Core Tasks}

The project will be a success if I have:
\begin{itemize}
  \item Successfully implemented a parser / import tool for the chosen corpus
    format.
	\item Successfully implemented a statistical model such as a multiple
		viewpoint system capable of generating melody.
	\item Successfully implemented a technique based on recurrent neural
		networks capable of generating melody.
  \item Performed quantitative evaluation of at least one of the models, using
    music-theoretic and/or information-theoretic measures.
  \item Performed a human evaluation and comparison of the two models. The
    success of this stage is not predicated on the answers to any questions
    posed in the evaluation, but merely whether the evaluation is conducted in a
    scientific manner, answering questions such as:
	\begin{itemize}
		\item Can human subjects distinguish the machine-composed output
			from the human-composed samples in the corpus?
		\item Do human subjects classify the machine-composed output as
			adhering to the specified style?
	\end{itemize}
  
  The evaluation will employ techniques to make scientific, \emph{falsifiable}
  claims concerning properties of the models and their output, e.g.\ as per
  Pearce et al.\ 
  \cite{pearce2001evaluation}.

	\end{itemize}

\subsection{Extension Tasks}

The project will be judged as a success if all the core tasks have been
completed. The extension tasks won't be used to judge the success of the
project, but it will have gone above and beyond expectations if one of them is
completed.

These possible extensions include:
\begin{itemize}
	\item Extending a multiple viewpoint system to generate polyphony.
	\item Extending a recurrent neural network to generate polyphony.
  \item Creating a web interface to one or both of the systems to allow end
    users to train and generate from their own models.
\end{itemize}

\newpage

\section{Timetable}

\begin{longtable}{ p{4cm} | p{11cm} } \hline 

06/10/16 -- 19/10/16 Michaelmas Weeks 1-2 & \textbf{Proposal}. Done. \\ \hline
20/10/16 -- 02/11/16 Michaelmas Weeks 3-4 & \textbf{Research Phase}. 
Research multiple viewpoint systems, RNNs, and perform preliminary research on
evaluation techniques.  Investigate options for corpus material and format.
Devise and fix internal representation for musical data. Prepare test corpus.
Gain familiarity with core libraries (e.g.\ TensorFlow) and modern C++ features.
\newline

\textbf{Milestone}: Report summarising research and design decisions to
supervisor. \emph{Due 2nd November 2016}.
\\ \hline
03/11/16 -- 16/11/16 Michaelmas Weeks 5-6 & \textbf{Implementation:
	Preliminaries}.
Implement parser for format determined during research phase (e.g.\ ABC, MIDI,
MusicXML). Implement internal representation and associated music-theoretic
tooling. Once complete, commence work on MVS implementation. \newline

\textbf{Milestone}: Working parser with accompanying (passing) tests. Underlying
context model implementation for MVS complete. \emph{Due 16th
  November 2016}.
\\ \hline

17/11/16 -- 30/11/16 Michaelmas Weeks 7-8 & \textbf{Implementation: MVS}.
Continue implementation of MVS. The system need not be complete in the exact
viewpoints used, but the underlying machinery should be. \newline

\textbf{Milestone}: Working MVS demonstrated to supervisor. Induction of
viewpoints, generation from individual viewpoints and interpolation of viewpoint
predictions should be demonstrated. Core functionality (e.g. context model
induction) should be accompanied by passing unit tests. 
\emph{Due 30th November 2016}. \\ \hline

01/12/16 -- 14/12/16 Xmas Vac.\ Weeks 1-2 & \textbf{Implementation: RNN}.
Begin implementation of RNN. Some slack allowed in this period. \newline

\textbf{Milestone}: Basic RNN implementation without LSTM, using a simple input
representation (e.g.\ pitch-only) completed. \newline 
\emph{Due 14th December 2016}.
\\ \hline

15/12/16 -- 04/01/17 Xmas Vac.\ Weeks 3-5 & \textbf{Implementation: MVS II}.
Refine implementation of MVS. Experiment with various different viewpoints.
Refer to Whorley \cite{whorley2013phd} for direction here.
\newline

\textbf{Extension}: If time, extend MVS to generate polyphony in this stage. 
\newline

\textbf{Milestone}: MVS assembled, document summarising implementation progress
including sample output sent to supervisor. 
\newline \emph{Due 4th January 2017}.
\\ \hline
05/01/17 -- 18/01/17 Xmas Vac.\ Weeks 6-7 & \textbf{Implementation: RNN II}.
Refine implementation of RNN. Investigate implementing LSTM model. Some slack
allowed in this phase. \newline

\textbf{Extension}: Continue extending MVS to generate polyphony, or extend
RNN. \newline

\textbf{Milestone}: Demonstration of RNN and MVS to supervisor for start of
term. \emph{Due 19th January 2017}.
\\ \hline

19/01/17 -- 01/02/17 Lent Weeks 1-2 & \textbf{Review}.
Review progress so far. Continue iteratively refining models. Design listening
trials for evaluation. \newline

\textbf{Milestone}: Progress Report \& Presentation. \emph{Due 12pm, 3rd
	February}.
\\ \hline

02/02/17 -- 15/02/17 Lent Weeks 3-4 & \textbf{Evaluation}.
Arrange and conduct listening trials. Also perform any ``offline'' evaluation in
this phase e.g.\ collecting data from models, analysing output etc. \newline

\textbf{Extension}: If time available, and requisite extension components
complete, carry out music-theoretic evaluation of polyphony. \newline

\textbf{Milestone}: Listening trials arranged \emph{by 9th February}.
\\ \hline

16/02/17 -- 01/03/17 Lent Weeks 5-6 & \textbf{Dissertation: Preparation}.
Finish evaluation and data collection. Write bullet-point conclusion and 
evaluation chapters. Write up implementation. \newline

\textbf{Milestone}: Data from evaluation collected by \emph{23rd February}.
Conclusion outlined in bullet points. Implementation chapter
complete. \emph{Due 1st March 2017}.
\\ \hline

02/03/17 -- 15/03/17 Lent Weeks 7-8 & \textbf{Dissertation: Introduction}.
Write preparation and introduction chapters of dissertation. \newline

\textbf{Milestone}: Chapter drafts to supervisor. \emph{Due 15th March}.
\\ \hline

16/03/17 -- 29/03/17 Easter Vac.\ Weeks 1-2 & Write evaluation chapter of
dissertation. Revise drafts of chapters written thus far. \newline

\textbf{Milestone}: First complete draft to supervisor. \emph{Due 29th March}.
\\ \hline

30/03/17 -- 12/04/17 Easter Vac.\ Weeks 3-4 & 
Applying supervisor's corrections to dissertation. Continue working on
dissertation (e.g.\ figures etc.)
\\ \hline

13/04/17 -- 26/04/17 Easter Vac.\ Weeks 5-6 & 
Finish the dissertation by the end of the Easter vacation. Slack for revision
allowed in this phase. \newline

\textbf{Milestone}: Final draft sent to supervisor by \emph{26th April 2017}.
\\ \hline
\end{longtable}

\printbibliography

\end{document}
