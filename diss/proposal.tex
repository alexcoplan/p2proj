% Note: this file can be compiled on its own, but is also included by
% diss.tex (using the docmute.sty package to ignore the preamble)
\documentclass[12pt,a4paper,twoside]{article}
\usepackage[UKenglish]{isodate}
\usepackage[pdfborder={0 0 0}]{hyperref}
\usepackage[margin=25mm]{geometry}
\usepackage{graphicx}
\usepackage{parskip}
\usepackage{enumitem}
% \usepackage{mathpazo}
% \usepackage{eulervm}
\usepackage{microtype}

\usepackage[style=numeric,backend=bibtex]{biblatex}
\bibliography{refs.bib}

\begin{document}

\cleanlookdateon

\begin{center}
\Large
Computer Science Tripos -- Part II -- Project Proposal\\[4mm]
\LARGE
A Comparison of Statistical Models and Recurrent Neural Networks Applied to the
Generation of Music\\[4mm]

\large
Alex Coplan, St Catharine's College

Originator: Alex Coplan

\today
\end{center}

\vspace{5mm}

\textbf{Project Supervisor:} 

\textbf{Director of Studies:} Dr S. Taraskin

\textbf{Project Overseers:} Dr M. Fiore \& Dr I. Leslie

% Main document

\section*{Introduction}

The goal of this project is to implement, evaluate, and compare two different
techniques for the algorithmic generation of music. I am particularly interested
in the generation of melody, and ultimately, polyphony. In particular, the two
classes of techniques I intend to consider for this project are:
\begin{itemize}[itemsep=0mm]
	\item Statistical history models such as \emph{multiple viewpoint
			systems} \cite{conklin1995viewpoints}.
	\item Recurrent neural networks.
\end{itemize}

The exact statistical model(s) to be investigated will be determined by the end
of the research phase of the project.

My intention is to undertake an investigation into the algorithmic composition
of polyphonic music. The first major problem that needs to be tackled in an
investigation into polyphonic composition is that of melody generation. Once
this problem has been addressed, one would subsequently consider the problem of
``given a melody (the \emph{subject}), compose a second (independent) melody
(the \emph{countersubject}) which interacts with, and is coherent with the
subject.'' Since the lines in polyphony should be \emph{independent} melodies,
it is necessary to approach the problem of melody generation first.

I shall follow the approach often taken in the literature of restricting the
domain of source material to stem from a particular musical idiom, e.g.\
\cite{pearce2001evaluation}. This is desirable for a number of reasons, not
least because it introduces a useful evaluation criterion: do the compositions
produced by the system exhibit a coherent musical style, consistent with that
exhibited by the material in the corpus?

I propose that the core of the project investigate the application of these
techniques to melody generation. As an optional extension, an investigation
could then be carried out into the composition of two-part polyphonic music,
using these techniques and/or extensions thereof.

In order for this project to be evaluated effectively, in addition to any
information-theoretic or music-theoretic analyses, it is necessary to perform
listening trials on human subjects. Pearce et al.\ \cite{pearce2001evaluation}
outline a framework for evaluation which allows more scientific claims to be
made as a result of the evaluation process. Evaluation would be performed in the
form of a blind trial where the subjects are asked to classify compositions as
human or machine-composed. In this work, it is noted that the participants
exhibited a bias towards classifying compositions as machine composed. This is
something that should be taken into account when designing the evaluation
methodology.
\subsection*{Background}

% A large volume of research has already been performed in
% computational \emph{homophonic} harmonisation, an extremely popular example
% being the harmonisation of Lutheran chorale melodies in the style of J.S. Bach,
% e.g.\ \cite{allan2005chorales}\cite{phon2002computational}\ignorespaces
% \cite{phon1999four}\cite{whorley2015improved}.
% 
% The problem of homophonic harmonisation is essentially ``given a melody, output
% a list of chords that accompany the melody''. 

Markov processes are natural statistical models for the analysis of melody, and
are well known as tools for composition \cite{ames1989markov}. Although
effective, Markov processes are far from perfect tools for modelling music.
Specifically, a basic pitch-duration Markov process disregards a considerable
amount of musical information available in the context
\cite{conklin1995viewpoints}.  

However, simply incorporating more musical features into the state space of a
Markov chain leads to an exponential blow-up in space complexity and
necessitates both a large amount of training data for good performance, as well
as solving the sparse data problem (\cite{conklin2003music}, section 2.1).
Moreover, Markov chains do not make use of the long-term context of a system,
which is necessary for modelling the broader sense of sequence and structure
which is present in music.

Conklin et al.\ \cite{conklin1995viewpoints} introduce the method of
\emph{multiple viewpoints} which uses the interpolation of the predictions of
many different context models, each of which considers a different musical
attribute (or some combination of attributes). In particular, a multiple
viewpoint system partitions these musical attributes (or \emph{types}) into
long-term and short-term types, with the system using a weighted combination of
the long-term and short-term predictions.

In this work, it is also conjectured that lower-entropy models generate music
which is closer in style to that of the corpus. Whorley et al.\
\cite{whorley2016music} have recently successfully used the technique of
multiple viewpoints along with information-theoretic analysis and search
techniques, demonstrating that lower entropy solutions correspond to improved
output quality.

Conklin \cite{conklin2003music} notes that it is (largely) not necessary to
distinguish between statistical music analysis and generation. Generation can be
thought of as sampling from the empirically-determined distribution. Moreover,
random walk, a common method for generating music from a Markov model, is not
necessarily the best method for sampling from the model's distribution, in that
it does not consistently yield compositions with high overall probability. In
this project, I would therefore also consider exploring different techniques for
sampling from statistical models.

It is well known that Recurrent Neural Networks (RNNs) can effectively generate
sequences. RNNs have seen more successful application in music following the
introduction of long short-term memory (LSTM) techniques \cite{eck2002lstm}.
Without use of LSTM, RNNs exhibit similar problems to Markov chains in that the
output does not contain the elements of sequence and structure that one might
expect from compositions in the corpus.

\subsection*{Motivation}

Algorithmic composition is of general interest in computational creatitvity, but
also has a number of practical applications; one such application being in
\emph{machine-assisted composition}, where a music generation technique would
aid a human composer by extending or generating musical ideas.

% TODO: mention about comparison between statistical models and RNNs?

\section*{Starting point}

In Lent term of 2016, I gave a talk (as part of Churchill college's Computer
Science talk series) on melody generation using Markov chains. I also
constructed a demo in Ruby which implemented a parser for ABC
notation\footnote{\url{http://abcnotation.com/}} along with a simple Markov
chain model, trained of a small corpus of hymn tunes, which generated tunes by
random walk. 

Although this experience lead me to this choice of project, I will not be using
any of the code from this in my project. The parser, internal representation and
music theory engine will be written from scratch in C++ or Java, and the neural
network will be implemented using a library such as Google's
TensorFlow\footnote{\url{https://www.tensorflow.org/}}.

As an organ scholar (and previously an A-Level music student), I have
considerable experience with performing polyphonic music (and some experience of
analysis), especially that of the renaissance and baroque eras. I believe this
domain knowledge will prove especially useful for making musically-informed
decisions in this project. 

\section*{Resources required}

For this project I shall primarily use my own laptop. Backup will be primarily
in the form of a GitHub-hosted repository, but I will also perform backups of
the project files to an external hard drive as well as multiple cloud providers
(Google, Apple, Dropbox) and the MCS. Should my main computer suddenly fail, I
can easily continue the project using MCS computers by cloning the code from the
GitHub repository.

Although datasets can easily be compiled from online sources, it may also be of
use to have a MIDI keyboard to be able to input arbitrary musical data. I own a
MIDI keyboard which would be suitable for these purposes. I will make use of
open-source software (such as MuseScore\footnote{\url{https://musescore.org/}})
for synthesis of MIDI and other musical data. I require no other special
resources.

\section*{Work to be done}

I will employ an agile software development methodology when undertaking this
project. The project breaks down into the following sub-projects:

\begin{enumerate}

\item Devising and implementing an internal representation of musical data,
	along with a simple ``music theory engine'' to process this data.  

\item Implementing a simple parser for some form of input notation (ABC, MIDI,
	MusicXML); the exact form to be determined in the research phase.  

\item Implementing the first iteration of the multiple viewpoint system.

\item Implementing the first iteration of the RNN for melody generation.  

\item Devising a scheme for human evaluation.

\item Making iterative improvements to the two models.

\item Carrying out the human evaluation.

\end{enumerate}

\section*{Success criteria}

\subsection*{Core Tasks}

The project will be a success if I have:
\begin{itemize}
	\item Successfully implemented a statistical model such as a multiple
		viewpoint system capable of generating melody.
	\item Successfully implemented a recurrent neural network capable of
		generating melody.
	\item Performed a meaningful comparison and evaluation of the two
		models, answering questions such as:
	\begin{itemize}
		\item Can human subjects distinguish the machine-composed output
			from the human-composed samples in the corpus?
		\item Do human subjects classify the machine-composed output as
			adhering to the specified style?
	\end{itemize}

	Note that the success of the evaluation stage is not predicated on the
	answers to the questions given above, but merely whether the evaluation
	is conducted in a scientific manner.
\end{itemize}

\subsection*{Extension Tasks}

The project will be judged as a success if all the core tasks have been
completed. The extension tasks won't be used to judge the success of the
project, but it will have gone above and beyond expectations if one of them is
completed.

These possible extensions include:
\begin{itemize}
	\item Extending a multiple viewpoint system to generate polyphony.
	\item Extending a RNN to generate polyphony.
	\item Exploring extensions and adaptations of multiple viewpoint
		systems.
\end{itemize}

\section*{Timetable}

Planned starting date is 16/10/2011.

Timetable to be completed in a later draft.

% \begin{enumerate}
% 
% \item \textbf{Michaelmas weeks 2--4} Learn to use X. Read book Y. Read papers Z.
% 
% \item \textbf{Michaelmas weeks 5--6} Do preliminary test of Q.
% 
% \item \textbf{Michaelmas weeks 7--8} Start implementation of main task A.
% 
% \item \textbf{Michaelmas vacation} Finish A and start main task B.
% 
% \item \textbf{Lent weeks 0--2} Write progress report. Generate corpus of test
% 	examples. Finish task B.
% 
% \item \textbf{Lent weeks 3--5} Run main experiments and achieve working project.
% 
% \item \textbf{Lent weeks 6--8} Second main deliverable here.
% 
% \item \textbf{Easter vacation:} Extensions and writing dissertation main
% 	chapters.
% 
% \item \textbf{Easter term 0--2:}  Further evaluation and complete dissertation.
% 
% \item \textbf{Easter term 3:} Proof reading and then an early submission so as
% 	to concentrate on examination revision.
% 
% \end{enumerate}

\printbibliography

\end{document}
