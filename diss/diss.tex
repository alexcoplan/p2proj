% Template for a Computer Science Tripos Part II project dissertation
\documentclass[12pt,a4paper,twoside,openright]{report}
\usepackage[UKenglish]{isodate}
\usepackage[pdfborder={0 0 0}]{hyperref}    % turns references into hyperlinks
\usepackage[margin=25mm]{geometry}  % adjusts page layout
\usepackage{graphicx}  % allows inclusion of PDF, PNG and JPG images
\usepackage{parskip}
\usepackage{verbatim}
\usepackage{color}
\usepackage{enumitem}
\usepackage{longtable}
\usepackage{microtype}

%\usepackage[style=numeric,backend=biber]{biblatex}
%\addbibresource{refs.bib}

\usepackage{docmute}   % only needed to allow inclusion of proposal.tex

\newcommand{\todo}{\textcolor{red}{\textbf{todo}~}}

\raggedbottom                           % try to avoid widows and orphans
\sloppy
\clubpenalty1000%
\widowpenalty1000%

\renewcommand{\baselinestretch}{1.1}    % adjust line spacing to make
                                        % more readable

\begin{document}

\cleanlookdateon

%%%%%%%%%%%%%%%%%%%%%%%%%%%%%%%%%%%%%%%%%%%%%%%%%%%%%%%%%%%%%%%%%%%%%%%%
% Title

\pagestyle{empty}

\rightline{\LARGE \textbf{Alex Coplan}}

\vspace*{60mm}
\begin{center}
\Huge
\textbf{A Comparison of Statistical Models and Recurrent Neural Networks for the
Generation of Music} \\[5mm]
Computer Science Tripos -- Part II \\[5mm]
St Catharine's College \\[5mm]
\today  % today's date
\end{center}

%%%%%%%%%%%%%%%%%%%%%%%%%%%%%%%%%%%%%%%%%%%%%%%%%%%%%%%%%%%%%%%%%%%%%%%%%%%%%%
% Proforma, table of contents and list of figures

\pagestyle{plain}

\chapter*{Proforma}

{\large
\begin{tabular}{r p{10.5cm}}
Name:               & \bf Alex Coplan                       \\
College:            & \bf St Catharine's College                     \\
Project Title:      & \bf A Comparison of Statistical Models and Recurrent
Neural Networks for the \newline Generation of Music \\
Examination:        & \bf Computer Science Tripos -- Part II, July 2017  \\
Word Count:         & \todo\footnote{1} \\
Project Originator: & Alex Coplan \\
Supervisor:         & Matthew Ireland                    \\ 
\end{tabular}
}
\footnotetext[1]{This word count was computed
by \texttt{detex diss.tex | tr -cd '0-9A-Za-z $\tt\backslash$n' | wc -w}
}
\stepcounter{footnote}


\section*{Original Aims of the Project}

The original aim of this project was to implement two models for music
generation and subsequently compare them: namely, a \emph{recurrent neural
network} and \emph{multiple viewpoint system}. The two models were
to be compared using both a listening survey involving human participants and
objective metrics of evaluation, such as information-theoretic measures
of predictive performance.

\section*{Work Completed}

All that has been completed appears in this dissertation.

\section*{Special Difficulties}

\todo
 
\newpage
\section*{Declaration}

I, Alex Coplan of St Catharine's College, being a candidate for Part II of the
Computer Science Tripos, hereby declare that this dissertation and the work
described in it are my own work, unaided except as may be specified below, and
that the dissertation does not contain material that has already been used to
any substantial extent for a comparable purpose.

\bigskip
\leftline{Signed }

\medskip
\leftline{Date }

\tableofcontents

\listoffigures

\newpage
\section*{Acknowledgements}

Acknowledge acknowledge acknowledge.

%%%%%%%%%%%%%%%%%%%%%%%%%%%%%%%%%%%%%%%%%%%%%%%%%%%%%%%%%%%%%%%%%%%%%%%
% now for the chapters

\pagestyle{headings}

\chapter{Introduction}

The modelling and automated generation of music is a central task in an approach
to understanding computational creativity. It is natural to ask whether
computers can produce music that is compelling to humans, and indeed, this
question has long been posed by researchers, with practical efforts dating back
to the mid-1950s \cite{ames1987automated}. 

This dissertation investigates the application of two modern techniques to
\emph{melody generation}, a task highly amenable to statistical
modelling and machine learning. 

An automated system for melody generation is motivated by end-user applications
in \emph{computer-assisted composition}, whereby the system can provide
inspiration for a human composer, either by generating entirely novel melodies
within stylistic constraints, or extending or extemporising based on melodic
fragments wrtiten by the human composer. Such a system might augment the
capabilities of typical music notation software. 

Markov modelling is a simple yet effective technique for capturing the
statistics of sequential data. An obvious tool to apply to the modelling of
melody is the Markov chain \cite{ames1989markov}. However, an important
observation to make is that music has a rich underlying structure: modelling the
statistics of notes directly (the \emph{surface structure}) is insufficient to
capture the complex language of musical style. This observation motivated the
development of more sophisticated models, known as \emph{multiple viewpoint
systems} \cite{conklin1995viewpoints}.  A multiple viewpoint system exploits the
rich underlying event structure of complex languages by combining the
predictions of an ensemble of Markov models, each modelling a different
attribute of an event space.

Recurrent neural networks (RNNs), the natural topology of neural network for
modelling sequential data, have been widely applied to tasks such as language
modelling, optical character recognition, time-series modelling and forecasting,
and indeed to the modelling of music (\todo cite). A RNN is an end-to-end
sequence learning tool which relies on very little domain knowledge aside from
the chosen input representation. This means that a system designed for
character-level language modelling can equally be trained to model melody
without changing the network architecture. In practice, however, we shall see
that slight architectural modifications can be of use.

The aim of this work is to compare the performance of multiple viewpoint systems
with that of recurrent neural networks on a task of stylistically-constrained
melody generation. In particular, we assess the predictive performance of the
models on a corpus of melodies used in the chorale harmonisations of J.S.\ Bach.
Furthermore, we compare the sampled outputs from each model by means of a
listening survey involving human participants.

\section{Background}

The development and implementation of the recurrent neural network in this work
can be understood with minimal knowledge of music theory. In general, the
multiple viewpoint formalism can also be expressed independently of any domain
knowledge.  However, to understand the application of the multiple viewpoint
framework to music requires an elementary understanding of music theory.  This
section introduces the relevant terminology to enable such dissussion.

We shall constrain our discussion to the context of Western classical music,
since the style which we wish to model lies within this context. A \emph{pitch}
is an abstract concept which is related to the frequency of a musical
note. Two pitches are said to be an \emph{octave} apart if one has double the
frequency of the other. In Western classical music, each octave is divided into
twelve distinct note names. A note name, such as G, is typically thought of as
not just a single pitch, but a \emph{pitch class}: a set constructed by taking
the transitive reflexive closure of the \emph{octave} relation.

The mapping between pitch and frequency is known as \emph{temperament}, which on
modern instruments is typically \emph{equal}, meaning that the frequency space is
equally divided among the twelve pitches in an octave. More precisely, the
$k$\textsuperscript{th} note of a chromatic scale with base frequency $\nu_0$ is
given by $2^{k/12}\cdot\nu_0$. As an aside, all recordings of
model outputs in this work were made using equal temperament. Herein
we will work with the abstraction of pitch, ignoring the underlying frequencies
of notes.

\todo SPN, grounding with A440.

Terms to define: \emph{interval}, \emph{pitch}, \emph{pitch class},
\emph{metre}, \emph{bar}, \ldots

\section{Work Accomplished}




\section{Related Work}

\todo Structure

\chapter{Preparation}

\todo

\chapter{Implementation}


\chapter{Evaluation}


\chapter{Conclusion}

%%%%%%%%%%%%%%%%%%%%%%%%%%%%%%%%%%%%%%%%%%%%%%%%%%%%%%%%%%%%%%%%%%%%%
% the bibliography
\addcontentsline{toc}{chapter}{Bibliography}
\todo
%\printbibliography

%%%%%%%%%%%%%%%%%%%%%%%%%%%%%%%%%%%%%%%%%%%%%%%%%%%%%%%%%%%%%%%%%%%%%
% the appendices
\appendix

\chapter{Project Proposal}

\todo uncomment me!
%% Note: this file can be compiled on its own, but is also included by
% diss.tex (using the docmute.sty package to ignore the preamble)
\documentclass[12pt,a4paper,twoside]{article}
\usepackage[UKenglish]{isodate}
\usepackage[pdfborder={0 0 0}]{hyperref}
\usepackage[margin=25mm]{geometry}
\usepackage{graphicx}
\usepackage{parskip}
\usepackage{enumitem}
% \usepackage{mathpazo}
% \usepackage{eulervm}
\usepackage{microtype}

\usepackage[style=numeric,backend=bibtex]{biblatex}
\bibliography{refs.bib}

\begin{document}

\cleanlookdateon

\begin{center}
\Large
Computer Science Tripos -- Part II -- Project Proposal\\[4mm]
\LARGE
A Comparison of Statistical Models and Recurrent Neural Networks Applied to the
Generation of Music\\[4mm]

\large
Alex Coplan, St Catharine's College

Originator: Alex Coplan

\today
\end{center}

\vspace{5mm}

\textbf{Project Supervisor:} Matthew Ireland

\textbf{Director of Studies:} Dr S. Taraskin

\textbf{Project Overseers:} Dr M. Fiore \& Dr I. Leslie

% Main document

\section*{Introduction}

The goal of this project is to implement, evaluate, and compare two different
techniques for the algorithmic generation of music. I am particularly interested
in the generation of melody, and ultimately, \emph{polyphony}: multiple
independent melodies interacting with each other in harmonic coherence. In
particular, the two classes of techniques I intend to consider for this project
are:
\begin{itemize}[itemsep=0mm]
	\item Statistical history models such as \emph{multiple viewpoint
			systems} \cite{conklin1995viewpoints}.
	\item Recurrent neural networks.
\end{itemize}

The exact statistical model(s) to be investigated will be determined by the end
of the research phase of the project.

Algorithmic composition is of general interest in computational creativity, but
also has a number of practical applications; one such application being in
\emph{machine-assisted composition}, where a music generation tool aids a human
composer by extending or generating musical ideas. Such tools would be used by a
wide variety of music practitioners.

My intention is to undertake an investigation into the algorithmic composition
of polyphonic music. The first major problem that needs to be tackled in such an
endeavour is that of melody generation. Once this problem has been addressed,
one would subsequently consider the problem ``given a melody (the
\emph{subject}), compose a second (independent) melody (the
\emph{countersubject}) which interacts with, and is coherent with the subject.''
Since the lines in polyphony should be \emph{independent} melodies, it is
necessary to approach the problem of melody generation first.

I therefore propose that the core of the project investigate the application of
these techniques to melody generation. As an optional extension, an
investigation could then be carried out into the composition of two-part
polyphonic music, using these techniques and/or extensions thereof.

I shall follow the approach often taken in the literature of restricting the
domain of source material to stem from a particular musical idiom, e.g.\
\cite{pearce2001evaluation}. This is desirable for a number of reasons, not
least because it introduces a useful evaluation criterion: do the compositions
produced by the system exhibit a coherent musical style, consistent with that
exhibited by the material in the corpus?

In order for this project to be evaluated effectively, in addition to any
information-theoretic or music-theoretic analyses, it is necessary to perform
listening trials on human subjects. Pearce et al.\ \cite{pearce2001evaluation}
outline a framework for evaluation which allows more scientific claims to be
made as a result of the evaluation process. Evaluation would be performed in the
form of a blind trial where the subjects are asked to classify compositions as
human or machine-composed. In this work, it is noted that the participants
exhibited a bias towards classifying compositions as machine composed. This is
something that should be taken into account when designing the evaluation
methodology. An avenue for investigation in this respect is the method of
\emph{three-alternative forced choice}.

Conklin \cite{conklin2003music} notes that random walk is not necessarily the
best method of sampling from a statistical distribution such as that of a
multiple viewpoint system or Markov chain. In this project, I would therefore
also consider exploring different techniques for sampling from statistical
models.
 
\subsection*{Background}

Markov processes are natural statistical models for the analysis of melody, and
are well known as tools for composition \cite{ames1989markov}. Although
effective, Markov processes are far from perfect tools for modelling music.
Specifically, a basic pitch-duration Markov process disregards a considerable
amount of musical information available in the context
\cite{conklin1995viewpoints}.  

However, simply incorporating more musical features into the state space of a
Markov chain leads to an exponential blow-up in space complexity and
necessitates both a large amount of training data for good performance, as well
as solving the sparse data problem (\cite{conklin2003music}, section 2.1).
Moreover, Markov chains do not make use of the long-term context of a system,
which is necessary for modelling the broader sense of sequence and structure
which is present in music.

Conklin et al.\ \cite{conklin1995viewpoints} introduce the method of
\emph{multiple viewpoints} which uses the interpolation of the predictions of
many different context models, each of which considers a different musical
attribute (or some combination of attributes). These include both short-term and
long-term attributes, enabling this method to capture sequence and structure.

It is well known that Recurrent Neural Networks (RNNs) can effectively generate
sequences. RNNs have seen more successful application in music following the
introduction of long short-term memory (LSTM) techniques \cite{eck2002lstm}.
Without use of LSTM, RNNs exhibit similar problems to Markov chains in that the
output does not contain the elements of sequence and structure that one might
expect from compositions in the corpus.

\section*{Starting point}

In Lent term of 2016, I gave a talk (as part of Churchill college's Computer
Science talk series) on melody generation using Markov chains. I also
constructed a demo in Ruby which implemented a parser for ABC
notation\footnote{\url{http://abcnotation.com/}} along with a simple Markov
chain model, trained of a small corpus of hymn tunes, which generated tunes by
random walk. 

Although this experience led me to this choice of project, the implementation of
a model such as a multiple viewpoint system is considerably more involved, and
the architecture vastly different. The implementation of this model will
therefore be carried out from scratch.  The neural network will be implemented
using a library such as Google's
TensorFlow\footnote{\url{https://www.tensorflow.org/}}.

As an organ scholar (and previously an A-Level music student), I have
considerable experience with performing polyphonic music (and some experience of
analysis), especially that of the renaissance and baroque eras. I believe this
domain knowledge will prove especially useful for making musically-informed
decisions in this project. 

\section*{Resources required}

For this project I shall primarily use my own laptop. Backup will be primarily
in the form of a GitHub-hosted repository, but I will also perform backups of
the project files to an external hard drive as well as multiple cloud providers
(Google, Apple, Dropbox) and the MCS. Should my main computer suddenly fail, I
can easily continue the project using MCS computers by cloning the code from the
GitHub repository.

Although datasets can easily be compiled from online sources, it may also be of
use to have a MIDI keyboard to be able to input arbitrary musical data. I own a
MIDI keyboard which would be suitable for these purposes. I will make use of
open-source software (such as MuseScore\footnote{\url{https://musescore.org/}})
for synthesis of MIDI and other musical data. I require no other special
resources.

\section*{Work to be done}

I will employ an agile software development methodology when undertaking this
project. The ordered list of sub-tasks within this project are:
\begin{enumerate}

\item Devising and implementing an internal representation of musical data,
	along with a simple ``music theory engine'' to process this data.  

\item Implementing a simple parser for some form of input notation (ABC, MIDI,
	MusicXML); the exact form to be determined in the research phase.  

\item Implementing and iteratively refining the statistical model (e.g. multiple
	viewpoint system).

\item Implementing and iteratively refining the RNN for melody generation.  

\item Designing and carrying out a scheme for human evaluation.

\item Making iterative improvements to the two models.

\end{enumerate}

\section*{Success criteria}

\subsection*{Core Tasks}

The project will be a success if I have:
\begin{itemize}
	\item Successfully implemented a statistical model such as a multiple
		viewpoint system capable of generating melody.
	\item Successfully implemented a technique based on recurrent neural
		networks capable of generating melody.
	\item Performed an evaluation and comparison of the two
		models, answering questions such as:
	\begin{itemize}
		\item Can human subjects distinguish the machine-composed output
			from the human-composed samples in the corpus?
		\item Do human subjects classify the machine-composed output as
			adhering to the specified style?
	\end{itemize}

	Note that the success of the evaluation stage is not predicated on the
	answers to the questions given above, but merely whether the evaluation
	is conducted in a scientific manner.
\end{itemize}

\subsection*{Extension Tasks}

The project will be judged as a success if all the core tasks have been
completed. The extension tasks won't be used to judge the success of the
project, but it will have gone above and beyond expectations if one of them is
completed.

These possible extensions include:
\begin{itemize}
	\item Extending a multiple viewpoint system to generate polyphony.
	\item Extending a RNN to generate polyphony.
	\item Exploring extensions and adaptations of multiple viewpoint
		systems.
\end{itemize}

\section*{Timetable}

Planned starting date is 16/10/2011.

\begin{tabular}{ p{4cm} | p{11cm} } \hline 
% TODO: 2 week blocks, start earlier to include proposal week.
% TODO: 2-page spread

16/10/16 - 27/10/16 Mich. Weeks 2-4 & \textbf{Research phase}.
Research multiple viewpoint systems, RNNs, and evaluation techniques.
Investigate options for corpus material and format. This will inform the type of
parser that should subsequently be implemented. Devise and fix an internal
representation for musical data. Design specific multiple viewpoint system based
on \cite{whorley2013phd}. The corpus should also be prepared as fully as
possible during this stage. Familiarisation with libraries (e.g. TensorFlow)
should be accomplished during this phase.

\textbf{Milestone}: Report summarising research handed to supervisor.
\\ \hline
03/11/16 - 10/11/16 \newline Mich. Week 5 & \textbf{Preliminary Implementation}. 
Implement parser for chosen corpus format. Implement internal representation as
determined in the previous phase. Write test suite for the above. 
\\ \hline
10/11/16 - 30/11/16 \newline Mich. Weeks 6-8 & \textbf{MVS Implementation I}.
First iteration of multiple viewpoint system to be completed in these three
weeks. System need not be complete in terms of the exact viewpoints used.
However, the underlying machinery should be. In particular, context models,
viewpoint representation, prediction interpolation etc. should all be
implemented, such that a MVS can be constructed, trained, and used for
generation.
\\ \hline
01/12/16 - 22/12/16 \newline Christmas Vac. Weeks 1-3 & \textbf{RNN Implementation I}.
Implement LSTM Recurrent Neural Network. 
\\ \hline
29/12/16 - 12/01/17 \newline Christmas Vac. Weeks 4-5 & \textbf{MVS Implementation II}.
Implement full multiple viewpoint system. Investigate different choices of
viewpoints as per the R\&D outlined in the research phase. 
\\ \hline
12/01/17 - 26/01/17 \newline Lent Week 0 & \textbf{RNN Implementation II}.
Finish RNN implementation. Train network on entire corpus.
\\ \hline

\end{tabular}

\printbibliography

\end{document}


\end{document}
